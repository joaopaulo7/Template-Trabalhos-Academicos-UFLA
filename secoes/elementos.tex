\chapter{\MakeUppercase{Utilizando a Classe no Formato da UFLA}}\label{sec:template}

Agora que já temos o conhecimento básico sobre como a linguagem \LaTeX\ funciona, podemos nos aprofundar nos detalhes de como utilizar esse template (e, especialmente, a classe \texttt{templufla}) para gerar documentos padronizados de alta qualidade.

\section{Seção secundária}\label{sec2:secoes}
\subsection{Seção terciária com texto extra comicamente grande para testar como será a quebra de linha do título}\label{sec3:teste}
\subsubsection{Seção quaternária}\label{sec4:teste}
\subsubsubsection{Seção quinária}\label{sec5:teste}

Um dos fatores fundamentais no desenvolvimento de um trabalho acadêmico é sua organização. Na normalização da UFLA e da ABNT, os trabalhos podem ter formato de livro, como esse template, mas devem ser divididos em seções, não em capítulos. Todo o texto pode ser dividido até a seção quinária.

No template, dispõem-se comandos para realizar tal divisão automaticamente, mas com uma importante observação: a seção primária utiliza o comando \verb|\chapter|. Isso foi feito para simplificar a migração de versões mais antigas e provavelmente será alterado no futuro. Abaixo estão listados os comandos de seccionamento disponíveis: 

\begin{alineas}
	\item \verb|\chapter|:  cria uma seção primária, que pode ser referenciada como \autoref{sec:template};
	\item \verb|\section|:  cria uma seção secundária, que pode ser referenciada como \autoref{sec2:secoes};
	\item \verb|\subsection|:  cria uma seção terciária, que pode ser referenciada como \autoref{sec3:teste};
	\item \verb|\subsubsection|:  cria uma seção quaternária, que pode ser referenciada como \autoref{sec4:teste};
	\item \verb|\subsubsubsection|:  cria uma seção quinária, que pode ser referenciada como \autoref{sec5:teste}.
\end{alineas}


\section{Alíneas e Subalíneas}
Segundo a normalização \cite{UFLA:2025}, ``as alíneas são usadas quando se deseja enumerar diversos assuntos de uma seção sem título próprio. Quando necessário, a alínea pode ser dividida em subalíneas.''

Nesse template, para listar elementos com alíneas, deve-se usar o ambiente \textbf{\texttt{alineas}}, que pode ser aninhado para criar subalíneas: 
\begin{lstlisting}[language={[LaTeX]TeX}]
	\begin{alineas}% [labelsep=0ex]
		\item item um;
		\item item dois:
		\begin{alineas} % subalíneas
			\item subitem 1;
			\item subitem 2.
		\end{alineas}
	\end{alineas}
\end{lstlisting}
Caso tenham letras duplicadas e queira mais espaço entre o indicador e o texto, basta adicionar a opção \texttt{[labelsep=0ex]} ao ambiente.

\vspace{18pt}
Abaixo é mostrado exemplo em texto retirado, na maior parte, do manual:
\begin{alineas}% adicione [labelsep=0ex] caso tenha letras duplicadas e queira mais espaço
	\item as alíneas são formadas pelos diversos assuntos que não possuem título próprio, dentro de uma mesma seção;
	\item o texto que antecede as alíneas termina em dois pontos;
	\item as alíneas devem ser indicadas alfabeticamente, em letra minúscula, seguida de parêntese. Utilizam-se letras dobradas, quando esgotadas as letras do alfabeto;
	\item as letras indicativas das alíneas devem apresentar recuo em relação à margem esquerda;
	\item o texto da alínea deve começar por letra minúscula e terminar em ponto e vírgula, exceto a última alínea que termina em ponto final;
	\item o texto da alínea deve terminar em dois pontos, se houver subalínea;
	\item a segunda e as seguintes linhas do texto da alínea começam sob a primeira letra do texto da própria alínea;
	\item sobre as subalíneas:
	\begin{alineas}
		\item as subalíneas devem começar por travessão seguido de espaço;
		\item as subalíneas devem apresentar recuo em relação à alínea;
		\item o texto da subalínea deve começar por letra minúscula e terminar em ponto e vírgula. A última subalínea deve terminar em ponto final, se não houver alínea subsequente;
		\item a segunda e as seguintes linhas do texto da subalínea começam sob a primeira letra do texto da própria subalínea;
		\item o recuo das margens também deve ser obedecido.
	\end{alineas}
\end{alineas}


\section{Figuras, Ilustrações e etc}

Utilizar figuras, ilustrações e elementos ``float'' no template é semelhante ao que foi explicado na \autoref{sec:latexFiguras}. Só é importante se atentar para dois detalhes: os títulos das figuras, que devem vir antes do \texttt{inludegraphics}, e a fonte, que deve ser posicionada após o \texttt{inludegraphics} e pode utilizar o comando pré definido \texttt{fonte}. Exemplo do código para a \autoref{fig:leaoCTAN}:
\begin{lstlisting}[language={[LaTeX]Tex}]
\begin{figure}
	\centering
	% \caption antes da figura
	\caption{Leão do site CTAN estudando \TeX} 
	\label{fig:leaoCTAN}
	\includegraphics[width=0.6\textwidth]{imgs/ctanlion}
	\fonte{Duane Bibby, diponibilizador por \url{www.ctan.org}}
\end{figure}
\end{lstlisting}

\begin{figure}[hb!]
	\centering
	\caption{Leão do site CTAN estudando \TeX}
	\label{fig:leaoCTAN}
	\includegraphics[width=0.6\textwidth]{imgs/ctanlion}
	\fonte{Duane Bibby, diponibilizado em \url{www.ctan.org/lion}}
\end{figure}




\section{Quadros e Tabelas}

Tabelas e quadros são elementos tabulares com conteúdos e formatos específicos. 

Tabelas envolvem principalmente números, utiliza-se linhas horizontais somente no topo, final e cabeçalhos e linhas verticais somente nos cabeçalhos. Para reduzir o tamanho do código dos cabeçalhos, o template dispõe do comando \texttt{linhadir}, que adiciona uma linha vertical à célula. Abaixo mostra-se um exemplo do código para a \autoref{tab:exemplo2}:

\begin{lstlisting}[language={[LaTeX]Tex}]
\begin{table}[h]
	\centering
	\caption{Exemplo de Tabela} 
	\label{tab:exemplo2}
	\begin{tabular}{c c c c }
		\hline
		\linhadir{Pessoa}& \linhadir{Livros}& \linhadir{Artigos}& Palestras\\
		\hline
		$p_1$& 1& 3& 4\\
		$p_2$& 1& 3& 3\\
		$p_3$& 1& 3& 4\\
		$p_4$& 3& 5& 2\\
		\hline 
	\end{tabular}
	\vspace{0.3cm}
	\fonte{original} %Fonte do tabela
\end{table}
\end{lstlisting}

\begin{table}[h]
	\centering
	\caption{Exemplo de Tabela} 
	\label{tab:exemplo2}
	\begin{tabular}{c c c c }
		\hline
		\linhadir{\bf Pessoa} & \linhadir{\bf Livros} & \linhadir{\bf Artigos} & \bf Palestras \\
		\hline
		$p_1$ & 1 & 3 & 4 \\
		$p_2$ & 1 & 3 & 3 \\
		$p_3$ & 1 & 3 & 4 \\
		$p_4$ & 3 & 5 & 2 \\
		\hline 
	\end{tabular}
	\vspace{0.3cm}
	\fonte{original} %Fonte do tabela
\end{table}


Quadros se diferem das tabelas por conterem principalmente dados textuais e suas células serem completamente fechadas. O \autoref{quad:exemplo} é um exemplo de quadro.

\begin{quadro}[h]
\centering
\caption{Opiniões sobre esse template}\label{quad:exemplo}
  \begin{tabular}{|l|p{9cm}|}
    \hline 
    \rowcolor[gray]{.9}
    \bf Nome& \bf Opinião\\
    \hline
    Jão& Desculpe, não posso comentar sobre isso.\\
    \hline
    Joana& Literalmente uma revolução do cinema nacional!\\
    \hline
    Jacquin& Esse autor é a vergonha da profisson!\\
    \hline
    Meu cachorro& Au! Au! Au! $\sim$Sons de papel sendo rasgado.\\
    \hline
    Overleaf& ASSINE, ASSINE O PREMIUM.\\
    \hline
    \end{tabular}
    
    \vspace{0.3cm}
	\fonte{original}
\end{quadro}


\section{Padrão das Referências}

Desde que o padrão descrito na \autoref{sec:Refs} seja seguido e partes importantes do template sejam mantidas, as referências não devem ser um problema. O template já vem configurado para o formato padrão da ABNT, desde que os arquivos \texttt{abntex.*}, o preâmbulo e a chamada dos comandos \texttt{bibliographystyle}, \texttt{citeoption}, \texttt{refencias} e \texttt{bibliography}, próximos ao final do arquivo principal, sejam mantidos.

É comum referências serem um problema e, apesar do \LaTeX\ ajudar muito nisso, o processo ainda pode ser complicado. O pacote \texttt{abntex2} está desatualizado, as correções precisaram ser \emph{hard-coded} e o arquivo principal reflete isso.

\textbf{Então, se você quer que as citações e referências sejam tão simples quanto adicionar um \texttt{bibtex} e usar o comando \texttt{cite}, EU TE SUPLICO, não altere o preâmbulo nem os comandos relacionados às referências no \texttt{tempulfa\_main.tex}.}

Referências sortidas para contribuir para a lista ao final (ignore): \cite{Eco1996,Booth2000,BIB2010,Hexsel2004,Franca2001,Gil2002,Porto2002,Silva2005,UFLA:2015,Moura1998,NBR6023:2002,LeGuin:1987}

