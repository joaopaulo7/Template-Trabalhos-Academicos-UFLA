\chapter{LISTAS, GLOSSÁRIO E ÍNDICE}\label{sec:listasEGlossario}

O template também inclui a criação dos elementos pré-textuais lista de abreviações (ou, como está no manual, abreviaturas), siglas e símbolos. Além disso, também inclui a criação do glossário e índice.

Todos esses novos elementos, exceto o índice, utilizam o mesmo pacote, \texttt{glossaries}, então, têm comportamento semelhante. Cada ítem deve ser adicionado aos glossários no preâmbulo e, após adicionados, podem ser referenciados no documento com o comando \texttt{gls}. Por exemplo:
	
\begin{figure}[!htb]
	\centering
	\caption{Inserindo ítem no \gls{glo:glossario} e o referindo no texto} %legenda
	\label{fig:exemploglossario1} %rotulo para refencia
	\begin{lstlisting}[language=tex]
\newglossaryentry{glo:glossario}{
	name={glossário}, 
	description={
		Relação de palavras ou expressões técnicas 
		de uso restrito ou de sentido obscuro, 
		utilizadas no texto, acompanhadas das 
		respectivas definições}
	}
		
%\makeglossaries % não é necessario nesse template pois já é chamado na classe
\begin{document}
...
\caption{Inserindo ítem no gls(glo:glossario) e o referindo no texto}
	\end{lstlisting}
	
	\fonte{original}
\end{figure}

O comando \textit{\textbf{gls}} escreve o nome do ítem quando é usado. Para que outro valor seja escrito, ou até para que nada seja escrtio, pode-se utilizar o comando \texttt{glslink}:
\begin{lstlisting}[language=tex]
	\glslink{<rótulo>}{<texto alternativo>}
\end{lstlisting}

É recomendável separar as adições aos glossários em arquivos para melhor organização e reduzir o tamanho do preâmbulo. Nesse template, cada glossário tem seu próprio arquivo, adicionado ao projeto via comandos \texttt{loadglsentries}, e os arquivos têm sua própria pasta. Essa organização pode ser alterada, desde que a mudança seja refletida no preâmbulo.

\newpage
Abaixo listamos o formato para a adição de ítens em cada glossário:

\begin{alineas}
	\item \textbf{abreviaturas}: adicionar uma abreviatura é simples e a sintaxe é relativamente fixa.\\ Um exemplo para a abreviatura de \gls{jan}:
		\begin{lstlisting}[language=tex]
\newabbreviation{jan} % rótulo
{jan.}% forma abreviada
{Janeiro} % forma completa
...
... Um exemplo para a abreviatura de \gls{jan}:
		\end{lstlisting}
	
	\item \textbf{siglas}: adicionar siglas é igualmente simples.\\ 
	Um exemplo para a sigla \gls{abnt}:
	\begin{lstlisting}[language=tex]
\newacronym{abnt} % rótulo
{ABNT} % sigla
{Associação Brasileira de Normas Técnicas} % nome completo
...
... Um exemplo para a sigla \gls{abnt}:
	\end{lstlisting}
	
	\item \textbf{símbolos}: adicionar símbolos é um pouco mais complicado, já que necessita de uma descrição. Como a lista é por ordem de uso, \gls{gama} deve aparecer antes de \gls{alfa}\\ 
	Um exemplo para o símbolo \gls{gama}:
	\begin{lstlisting}[language=tex]
\newglossaryentry{gama}{
	name={$\gamma$}, % o símbolo em questão
	description={Um número gama}, % descrição do símbolo
	type={symbols}} % indicador que é um símbolo
...
... Um exemplo para o símbolo \gls{gama}:
	\end{lstlisting}
	
	\item \textbf{glossário}: adicionar termos no glossário é igual ao exemplo mostrado antes.\\
	Nesse caso, no entanto, adicionamos uma \emph{tag} \LaTeX\ antes do nome, então temos que adicionar o valor "\emph{sort}", para que a ordenação seja correta.
	Um exemplo para o termo \gls{llm}:
	\begin{lstlisting}[language=tex]
\newglossaryentry{llm}{
	name={\emph{large language model}}, % o nome do termo
	description={Grande model de 
		linguagem. Ex:DeepSeek-R1}, % descrição do termo
	sort={large language model}} % chave para ordenação
...
... Um exemplo para o termo \gls{llm}:
	\end{lstlisting}
	
	\item \textbf{índice}: adicionar termos ao índice é o mais fácil de todos. Para adicionar um índice, basta utilizar o comando \emph{index}. Caso algum índice tiver um pai(ou mãe), basta adicionar o delimitador ``!''.\\
	Um exemplo para os termos conjunto\index{Conjunto}, conjunto aberto\index{Conjunto!aberto} e conjunto fechado\index{Conjunto!fechado}:
	\begin{lstlisting}[language=tex]
...Um exemplo para os termos conjunto\index{Conjunto},
conjunto aberto\index{Conjunto!aberto} e conjunto 
fechado\index{Conjunto!fechado}:
	\end{lstlisting}
	
\end{alineas}

Para mais informações sobre como usar esses pacotes, consulte as documentações oficiais do \texttt{glossaries}\footnote{https://ctan.org/pkg/glossaries?lang=en} e do \texttt{imakeidx}\footnote{https://ctan.org/pkg/imakeidx}.
