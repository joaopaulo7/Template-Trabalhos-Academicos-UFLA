\chapter{\MakeUppercase{Introdução}}

O objetivo deste documento é apresentar o uso básico da classe \texttt{templufla} para a elaboração de trabalhos acadêmicos da \glsxtrfull{ufla}, utilizando a linguagem de marcação \LaTeX\ \cite[ p.17]{Lamport1994}. A maioria dos comandos (macros) e ambientes das classes básicas da linguagem é válida também nessa classe, que é estendida com comandos para confecção da capa, páginas de rosto, dedicatórias, etc. A classe atual foi baseada na classe \texttt{uflamon}, disponível em CC-BY 4.0, no site Overleaf.

Inicialmente, a classe \texttt{uflamon} foi criada de acordo com as normas da PRPG/UFLA para produção de TCC \cite{PRPG2006}. Essas normas foram posteriormente atualizadas, de maneira geral pela UFLA, para a produção de monografias, dissertações e teses \cite{BIB2010}. A classe \texttt{uflamon} foi mantida até a 2ª edição da normalização da \gls{ufla} \cite{UFLA:2015}.

Então, em 2025, com a publicação da 6ª edição da normalização \cite{UFLA:2025}, os atuais mantenedores criaram a classe \texttt{templufla} e o atual template, adaptando o antigo código para a nova norma, corrigindo inadequações, atualizando pacotes e ampliando o formato.

Este texto, com o objetivo de familiarizar o usuário da linguagem \LaTeX\ e demonstrar o uso da classe \texttt{templufla}, encontra-se organizado da seguinte maneira: 

A \autoref{sec:latex} apresenta conceitos básicos da linguagem \LaTeX, servindo como um ponto de início para novos usuários. 
A \autoref{sec:template} demonstra como utilizar os elementos da classe para a criação de trabalhos no formato da \gls{ufla}.
A \autoref{sec:listasEGlossario} é um tutorial básico de como usar os pacotes \texttt{glossaries} e \texttt{imakeidx} para criar listas, glossário e índice. 
A \autoref{sec:conclusao} apresenta comentários e observações finais.
Por fim, os Anexos \ref{anex:lorem1} e \ref{anex:lorem2} demonstram o uso dos anexos e o \autoref{apen:apendice} mostra como elaborar um apêndice simples.