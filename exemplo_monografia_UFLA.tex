\documentclass{templufla}          % classe base para a monografia

%==============================================================================
% Utilizacao de pacotes
\usepackage[T1]{fontenc}         % usa fontes postscript com acentos
\usepackage[brazil]{babel}       % hifenização e títulos em português do Brasil
\usepackage[utf8]{inputenc}     % permite edição direta com acentos
\usepackage{amsmath}             % pacote da AMS para Matemática Avançada
\usepackage{amssymb}             % símbolos extras da AMS
\usepackage{latexsym}            % símbolos extras do LaTeX
\usepackage{graphicx}            % para inserção de gráficos
\usepackage{listings}            % para inserção de código
\usepackage{fancyvrb}            % para inserção de saídas de comandos
%\usepackage{enumerate}           % para personalizar lista enumeradas 
											%(incluso na classe)
\usepackage{longtable}           % para tambelas muito grandes NOVO!!!!

\usepackage{colortbl} % cores em tabelas
\newcolumntype{Z}{|>{\columncolor[gray]{0.9}}l|} %cor cinza em células
%\usepackage{array} % já incluso na classe
\newcolumntype{L}[1]{>{\raggedright\let\newline\\\arraybackslash\hspace{0pt}}m{#1}}
\newcolumntype{C}[1]{>{\centering\let\newline\\\arraybackslash\hspace{0pt}}m{#1}}
\newcolumntype{R}[1]{>{\raggedleft\let\newline\\\arraybackslash\hspace{0pt}}m{#1}}
\usepackage{multirow} % para juntar duas linhas em uma só

\usepackage{multicol} % para uso de várias colunas

% cores para os links cruzados
\usepackage{xcolor}
\definecolor{rltred}{rgb}{0.2,0,0}
\definecolor{rltgreen}{rgb}{0,0.2,0}
\definecolor{rltblue}{rgb}{0,0,0.2}

\usepackage[colorlinks=true,
            urlcolor=rltblue,       % \href{...}{...} external (URL)
            filecolor=rltgreen,     % \href{...} local file
            linkcolor=rltred,       % \ref{...} and \pageref{...}
            citecolor=rltgreen,
            pdftitle={Template de Trabalhos acadêmcios UFLA},
          pdfauthor={João Autor},
          pdfsubject={Template de Trabalhos acadêmcios UFLA.},
          pdfkeywords={Comunicação Científica. 2. Pesquisa . 3. Pesquisa Científica. 
 					 4. Redação. 5. Monografia.}%
]{hyperref} % para referência cruzadas
%\usepackage{hyperref}            % para referência cruzadas
\usepackage{subfigure}           % figuras dentro de figuras
\usepackage{caption}            % remodelando o formato dos títulos de 
                                 % tabelas e figuras

% configuração padrão do listings   
\lstset{
   language=Java,
   extendedchars=true,
   tabsize=3,
   basicstyle=\footnotesize\ttfamily,
   stringstyle=\em,
   showstringspaces=false ,
   inputencoding=utf8,
   literate={á}{{\'a}}1 {é}{{\'e}}1 {í}{{\'i}}1 {ó}{{\'o}}1 {ú}{{\'u}}1
   			{Á}{{\'A}}1 {É}{{\'E}}1 {Í}{{\'I}}1 {Ó}{{\'O}}1 {Ú}{{\'U}}1
   			{â}{{\^a}}1 {ê}{{\^e}}1 {ô}{{\^o}}1
   			{Â}{{\^A}}1 {Ê}{{\^E}}1 {Ô}{{\^O}}1
            {ã}{{\~a}}1 {õ}{{\~o}}1 {ẽ}{{\~e}}1 
            {ç}{{\c{c}}}1
            {à}{{\`{a}}}1{À}{{\`{A}}}1,
}

% para referências de acordo com a ABNT
% precisa instalar o abntex2 antes!!!
% http://abntex.codigolivre.org.br/
% comente se pretende usar outro padrão

%abnt-emphasize=bf coloca o título das bibliografias em negrito
%abnt-thesis-year=both
\usepackage[alf, abnt-etal-list=3, abnt-url-package=url, abnt-emphasize=bf]{abntex2cite}


% evite usar o hyperref com abntex, pode dar caca em urls... no linha anterior, informo
% para incluir urls usando o pacote url e não o hyperref
%
% caso queira o hyperref com abntex, comente a linha anterior e descomente a seguinte
%\usepackage[alf,abnt-etal-cite=3,abnt-etal-list=0,abnt-etal-text=emph]{abntex2cite}
%
% caso vc ainda use a versão anterior da abntex, comente a linha incluindo o abntex2cite
% e descomente a próxima linha 
%\usepackage[alf,abnt-etal-cite=3,abnt-etal-list=0,abnt-etal-text=emph]{abntcite}


%==============================================================================
% criação do glossário, listas de siglas, símbolos e etc
\loadglsentries{glossarios/abreviaturas}
\loadglsentries{glossarios/glossario}
\loadglsentries{glossarios/siglas}
\loadglsentries{glossarios/simbolos}
%==============================================================================

%==============================================================================
% para os fãs do Word, descomente as linhas abaixo
%\sloppy %mais espaço entre as linhas
%\usepackage{identfirst} %identando-se a primeira linha de cada seção
%\noindentfirst % Tire o comentário para manter o padrão do LaTeX.

%==============================================================================
% definido comandos na monografia - não é necessário na sua monografia 
% apenas para exemplificar a definição de novos comandos
\newcommand{\defs}[1]{\textsl{#1}}


% Especificando hifenizações que por ventura LaTeX não saiba fazer
% Por padrão 99,9% dos termos em português devem ser hifenizados corretamente.
\hyphenation{hardware software Li-nux am-bien-te diag-nos-ti-car coor-de-na-ção 
FAE-PE Recovery TelEduc Williams UFLA}

%==============================================================================
% Dados da monografia, capa: autor, titulo, banca, etc... - SUBSTITUA DE ACORDO
%==============================================================================
\author{Joana Autora da Silva}
\title{Template de Exemplo do Estilo Padrão UFLA}
\subtitle{Usando Latex! =)}
\engtitle{Example Template for UFLAS' Standard Style}
\engsubtitle{Using \LaTeX =)}
\edicao{6$^a$ edição atualizada, ampliada e revista}
\date{2025}
\tipo{Tese apresentada à Universidade Federal de Lavras, como parte das exigências do Programa de Pós-Graduação em Monografia, área de concentração em TCC, para a obtenção do título de Doutor.}
% use \orientador ou \orientadora quando for o caso
\orientador{Prof. DSc. José Orientador}
%\orientadora{}
% use \coorientador ou \coorientadora quando for o caso
\coorientadora{Prof. DSc. Maria Orientadora } % comente se não tiver coorientador
%\coorientador{}
\local{Lavras -- MG}
\bancaum{Prof. MSc. Antônio Banca Um}{UFM}
\bancadois{Prof. DSc. João Banca Dois}{FCO} % comente se sua banca tiver só um professor
\bancatres{Profa. Esp. Eliza Banca Três}{BELMIS}
\bancaquatro{Prof. Esp. Zorbison Uberplax IV}{UFO}
\defesa{11 de abril de 2025}
%==============================================================================
%##################################################
% Dados para Ficha catalográfica, gerada pelo sistema da Biblioteca da UFLA
% http://www.biblioteca.ufla.br/FichaCatalografica/
% dados para ficha catalográfica
% Elaboração da Ficha Catalográfica
\preparofichacat{Ficha catalográfica elaborada pela Coordenadoria de Processos Técnicos da Biblioteca Universitária da UFLA, com dados informados pelo(a) próprio(a) autor(a).}
% primeiro autor - como na primeira linha da ficha catalográfica
\fcautor{Silva, Joana Autora da}
% autores, separados por vírgula - na ficha catalográfica, no formato que
% vem após o título e a barra ("/")
\fcautores{Joana Autora da Silva, João Autor do Silvo}
% caso trabalho seja ilustrado (figuras, gráficos, tabelas, etc.), 
% então informar por meio do comando a seguir
% caso não seja ilustrado, basta comentá-lo
\fcilustrado{il.}
% dados da edição para a ficha 
\fcedicao{6$^a$ ed. rev., atual. e ampl.}
% tipo do trabalho (tese, dissertação, etc.), de acordo com sistema
% de geração de ficha catalográfica
\fctipo{Tese(doutorado)}
% ano da defesa, só precisa informar se for diferente do ano da publicação
% se forem iguais, comente a linha a seguir
\fcdatadefesa{2025}
% preencher aqui com os dados de catalogação gerados pelo sistema
\fccatalogacao{1. TCC. 2. Monografia. 3. Dissertação. 4. Tese. 5. Trabalho Científico – Normas. I. Universidade Federal de Lavras. II. Título.}
\fcclasi{808.066}

%##################################################

%\antesfichacat{\noindent Para citar este documento: \\UNIVERSIDADE FEDERAL DE LAVRAS. Biblioteca Universitária. \textbf{Manual de normalização e estrutura de trabalhos acadêmicos: TCC, monografias, dissertações e teses}. 2. ed. rev., atual. e ampl. Lavras, 2015. Disponível em: \url{http://www.biblioteca.ufla.br/wordpress/wpcontent/uploads/bdtd/manual_normalizacao_UFLA.pdf}. Acesso em: data de acesso.}

%\depoisfichacat{\noindent A reprodução e a divulgação total ou parcial deste trabalho são autorizadas, por qualquer meio convencional ou eletrônico, para fins de estudo e pesquisa, desde que citada a fonte.\\
%\newline
%{\small Este documento possui páginas em branco para facilitar a impressão frente-e-verso.}}

%##################################################

%##################################################

% para os exemplos do manual
%\newenvironment{exemplomanual}{
%\vspace{0.5cm}
%\noindent\begin{minipage}{\textwidth}
%\noindent\rule{\textwidth}{0.5pt}
%\vspace{-1cm}
%\begin{flushleft}
%}{
%\end{flushleft}
%\vspace{-0.6cm}
%\noindent\rule{\textwidth}{0.5pt}
%\vspace{0.3cm}
%\end{minipage}
%}

%\newenvironment{exemplomanuallista}{
%\vspace{0.3cm}
%\noindent\begin{minipage}{\textwidth - 0.5cm}
%\noindent\rule{\textwidth}{0.5pt}
%\vspace{-1cm}
%\begin{flushleft}
%}{
%\end{flushleft}
%\vspace{-0.6cm}
%\noindent\rule{\textwidth}{0.5pt}
%\vspace{0.3cm}
%\end{minipage}
%}

% por conta de alguns exemplos
%\usepackage{setspace}

%##################################################

% se vc já defendeu e tem o arquivo escaneado da folha de rosto, 
% descomente e altere o nome do arquivo
%\folhaAprovacaoAssinada{folharosto}

% Aqui começa o documento propriamente dito
\begin{document}

\maketitle

\dedic{Espaço reservado à dedicatória.}     % Dedicatórias\\

\thanks{Espaço reservado aos agradecimentos.}         % Agradecimentos

\epigrafe{ % citação opcional
Espaço reservado à epígrafe.\\
(Autor Desconhecido)}


% palavras-chave
\palchaves{Resumo; Palavras; Representativas.}
\resumo{Mudanças com relação ao original (Uflamon 3a Edição):\\
> Alteração para fonte do texto 12pt, folha a4 e onesided;\\
> Alteração no tamanho das margens;\\
> Adição dos indicadores de impacto;\\
> Coorientador(a) agora aparece na ficha catalográfica;\\
> Nome da/do autor(a) mais alto na capa;\\ 
> Adição de glossários, listas de termos e índices;\\
> Remoção de páginas duplas dos elementos pré-textuais;\\
> Remoção de <> entre links na referências;\\
> Reorganização dos arquivos;\\
> Uso de itálico nas expressões em latim nas referências (ex: \textit{et al} e [\textit{s.n.}]);\\
> Alteração na capitalização de citações de pessoas físicas (de AUTOR para Autor);\\
> Remoção do salto de linha entre o título e subtítulo;\\
> Suporte para mais de 26 anexos/apêndices (abençoado seja quem usar isso);\\
> Agora as listas de figuras e etc mantêm a formatação para qualquer índice (antes, figuras/tabelas > 10 quebravam a formatação);\\
> Agora o sumário mantém a formatação para qualquer índice (antes, sec/subsec > 10 quebravam a formatação);\\
> Logo da UFLA agora é em formato vetorial (resolução "infinita");\\
> Alteração de Large de 14.4pt para 14pt;\\
> Alteração de small de 10.95pt para 11pt;\\
> Uso de 12pt para títulos de figuras e tabelas;\\
> Uso de 11pt para notas, fichas e rodapés;\\
> Ajustes de posicionamento necessário graças às diferenças de tamanho das fontes e das margens;\\
> Adicionado suporte para acentos e ç nos listings;\\
\\
Se encontrar mais mudanças necessárias, por favor informar!\\
Contato:\\
https://github.com/joaopaulo7/Template-Trabalhos-Academicos-UFLA (preferêncial)\\
joao.lima10@estudante.ufla.br\\
}  % Resumo (digite aqui o resumo)

% keywords devem vir antes do abstract
\keywords{Summary; Words; Representative.} % keywords
\abstract{The abstract should contain representative words of the work content, located below the abstract, separated by two spaces, preceded by the keyword expression. These representative words are spelled with the first letter capitalized, separated by point.}


\indicadoresdeimpacto{De caráter institucional, é um item obrigatório como parte das exigências das PRPG/PROEC para os trabalhos de pós-graduação Stricto sensu da UFLA. O autor deve apresentar um relato dos impactos sociais, tecnológicos, econômicos e/ou culturais dos resultados obtidos, considerando as populações, sociedade e territórios, deixando evidente se esses impactos foram concretos e diretos ou em potencial. Ao elaborar o item sobre impactos, é importante: 
a) caracterizar e quantificar resultados dos impactos sociais, tecnológicos, econômicos e/ou culturais da melhor forma possível; 
b) estabelecer se há algum caráter extensionista no trabalho, demonstrando impacto e participação da sociedade externa à UFLA, como parceiros e público-alvo; 
c) definir o território e grupos populacionais impactados; 
d) quando possível, declarar público diretamente beneficiado e número de docentes, estudantes e técnicos envolvidos nas ações extensionistas; 
e) estabelecer em quais das oito áreas temáticas da Política Nacional de Extensão podem ser classificados os impactos do trabalho. São elas: 1 - Comunicação, 2 - Cultura, 3 - Direitos humanos e justiça, 4 - Educação, 5 - Meio ambiente, 6 - Saúde, 7 - Tecnologia e produção e 8 - Trabalho;
 f) demonstrar quais os impactos da pesquisa e se estão alinhados com os 17 (dezessete) Objetivos de Desenvolvimento Sustentável (ODS) da Organização das Nações Unidas (ONU).}

\impactindicators{Of an institutional nature, it is a mandatory item as part of the PRPG/PROEC requirements for UFLA's Stricto sensu postgraduate work. The author must present a report on the social, technological, economic and/or cultural impacts of the results obtained, considering the populations, society and territories, making it clear whether these impacts were concrete and direct or potential. When preparing the item on impacts, it is important to:
a) characterize and quantify the results of the social, technological, economic and/or cultural impacts in the best possible way;
b) establish whether there is any extensionist character to the work, demonstrating the impact and participation of society outside UFLA, such as partners and target audience;
c) define the territory and population groups impacted;
d) when possible, declare the public directly benefited and the number of teachers, students and technicians involved in the extension actions;
e) establish in which of the eight thematic areas of the National Extension Policy the impacts of the work can be classified. They are: 1 - Communication, 2 - Culture, 3 - Human rights and justice, 4 - Education, 5 - Environment, 6 - Health, 7 - Technology and production and 8 - Work;
f) demonstrate the impacts of the research and whether they are aligned with the 17 (seventeen) Sustainable Development Goals (SDGs) of the United Nations (UN).}

%##################################################

% Dados do guia
%\begin{titlepage}
%\pagestyle{empty}
%\renewcommand{\baselinestretch}{1}
%\enlargethispage{1.5cm}
%\input{reitoria}
%\cleardoublepage
%\end{titlepage}

%##################################################

% descomente para habilitar a lista desejada
\listoffigures                             % Lista de Figuras
%\listofilustracoes
%\listofgraficos						   % Lista de Gráficos
\listoftables                              % Lista de Tabelas
\listofquadros							   % Lista de Quadros
%\listofexemplos
%\listofteoremas
\listofabreviaturas
\listofsiglas
\listofsimbolos
\tableofcontents                           % Sumário

\clearpage

\pagestyle{ufla}

%==============================================================================
% incluindo as seções
\chapter{Introdução}

O objetivo deste documento é apresentar o uso básico da classe \texttt{templufla} para a elaboração de trabalhos acadêmicos da \glsxtrfull{ufla}, utilizando a linguagem de marcação \LaTeX\ \cite{Lamport1994}. A maioria dos comandos (macros) e ambientes das classes básicas da linguagem é válida também nessa classe, que é estendida com comandos para confecção da capa, páginas de rosto, dedicatórias, etc. A classe atual foi baseada na classe \texttt{uflamon}, disponível em CC-BY 4.0, no site Overleaf.

Inicialmente, a classe \texttt{uflamon} foi criada de acordo com as normas da PRPG/UFLA para produção de TCC \cite{PRPG2006}. Essas normas foram posteriormente atualizadas, de maneira geral pela UFLA, para a produção de monografias, dissertações e teses \cite{BIB2010}. A classe \texttt{uflamon} foi mantida até a 2ª edição da normalização da \gls{ufla} \cite{UFLA:2015}.

Então, em 2025, com a publicação da 6ª edição da normalização \cite{UFLA:2025}, os atuais mantenedores criaram a classe \texttt{templufla} e o atual template, adaptando o antigo código para a nova norma, corrigindo inadequações, atualizando pacotes e ampliando o formato.

Este texto, com o objetivo de familiarizar o usuário da linguagem \LaTeX\ e demonstrar o uso da classe \texttt{templufla}, encontra-se organizado da seguinte maneira: 

A \autoref{sec:latex} apresenta conceitos básicos da linguagem \LaTeX, servindo como um ponto de início para novos usuários. 
A \autoref{sec:template} demonstra como utilizar os elementos da classe para a criação de trabalhos no formato da \gls{ufla}.
A \autoref{sec:listasEGlossario} é um tutorial básico de como usar os pacotes \texttt{glossaries} e \texttt{imakeidx} para criar listas, glossário e índice. 
A \autoref{sec:conclusao} apresenta comentários e observações finais.
Por fim, os Anexos \ref{anex:lorem1} e \ref{anex:lorem2} demonstram o uso dos anexos e o \autoref{apen:apendice} mostra como elaborar um apêndice simples.
\chapter{\MakeUppercase{Utilizando a Classe no Formato da UFLA}}\label{sec:template}

Agora que já temos o conhecimento básico sobre como a linguagem \LaTeX\ funciona, podemos nos aprofundar nos detalhes de como utilizar esse template (e, especialmente, a classe \texttt{templufla}) para gerar documentos padronizados de alta qualidade.

\section{Seção secundária}\label{sec2:secoes}
\subsection{Seção terciária com texto extra comicamente grande para testar como será a quebra de linha do título}\label{sec3:teste}
\subsubsection{Seção quaternária}\label{sec4:teste}
\subsubsubsection{Seção quinária}\label{sec5:teste}

Um dos fatores fundamentais no desenvolvimento de um trabalho acadêmico é sua organização. Na normalização da UFLA e da ABNT, os trabalhos podem ter formato de livro, como esse template, mas devem ser divididos em seções, não em capítulos. Todo o texto pode ser dividido até a seção quinária.

No template, dispõem-se comandos para realizar tal divisão automaticamente, mas com uma importante observação: a seção primária utiliza o comando \verb|\chapter|. Isso foi feito para simplificar a migração de versões mais antigas e provavelmente será alterado no futuro. Abaixo estão listados os comandos de seccionamento disponíveis: 

\begin{alineas}
	\item \verb|\chapter|:  cria uma seção primária, que pode ser referenciada como \autoref{sec:template};
	\item \verb|\section|:  cria uma seção secundária, que pode ser referenciada como \autoref{sec2:secoes};
	\item \verb|\subsection|:  cria uma seção terciária, que pode ser referenciada como \autoref{sec3:teste};
	\item \verb|\subsubsection|:  cria uma seção quaternária, que pode ser referenciada como \autoref{sec4:teste};
	\item \verb|\subsubsubsection|:  cria uma seção quinária, que pode ser referenciada como \autoref{sec5:teste}.
\end{alineas}


\section{Alíneas e Subalíneas}
Segundo a normalização \cite{UFLA:2025}, ``as alíneas são usadas quando se deseja enumerar diversos assuntos de uma seção sem título próprio. Quando necessário, a alínea pode ser dividida em subalíneas.''

Nesse template, para listar elementos com alíneas, deve-se usar o ambiente \textbf{\texttt{alineas}}, que pode ser aninhado para criar subalíneas: 
\begin{lstlisting}[language={[LaTeX]TeX}]
	\begin{alineas}% [labelsep=0ex]
		\item item um;
		\item item dois:
		\begin{alineas} % subalíneas
			\item subitem 1;
			\item subitem 2.
		\end{alineas}
	\end{alineas}
\end{lstlisting}
Caso tenham letras duplicadas e queira mais espaço entre o indicador e o texto, basta adicionar a opção \texttt{[labelsep=0ex]} ao ambiente.

\vspace{18pt}
Abaixo é mostrado exemplo em texto retirado, na maior parte, do manual:
\begin{alineas}% adicione [labelsep=0ex] caso tenha letras duplicadas e queira mais espaço
	\item as alíneas são formadas pelos diversos assuntos que não possuem título próprio, dentro de uma mesma seção;
	\item o texto que antecede as alíneas termina em dois pontos;
	\item as alíneas devem ser indicadas alfabeticamente, em letra minúscula, seguida de parêntese. Utilizam-se letras dobradas, quando esgotadas as letras do alfabeto;
	\item as letras indicativas das alíneas devem apresentar recuo em relação à margem esquerda;
	\item o texto da alínea deve começar por letra minúscula e terminar em ponto e vírgula, exceto a última alínea que termina em ponto final;
	\item o texto da alínea deve terminar em dois pontos, se houver subalínea;
	\item a segunda e as seguintes linhas do texto da alínea começam sob a primeira letra do texto da própria alínea;
	\item sobre as subalíneas:
	\begin{alineas}
		\item as subalíneas devem começar por travessão seguido de espaço;
		\item as subalíneas devem apresentar recuo em relação à alínea;
		\item o texto da subalínea deve começar por letra minúscula e terminar em ponto e vírgula. A última subalínea deve terminar em ponto final, se não houver alínea subsequente;
		\item a segunda e as seguintes linhas do texto da subalínea começam sob a primeira letra do texto da própria subalínea;
		\item o recuo das margens também deve ser obedecido.
	\end{alineas}
\end{alineas}


\section{Figuras, Ilustrações e etc}

Utilizar figuras, ilustrações e elementos ``float'' no template é semelhante ao que foi explicado na \autoref{sec:latexFiguras}. Só é importante se atentar para dois detalhes: os títulos das figuras, que devem vir antes do \texttt{inludegraphics}, e a fonte, que deve ser posicionada após o \texttt{inludegraphics} e pode utilizar o comando pré definido \texttt{fonte}. Exemplo do código para a \autoref{fig:leaoCTAN}:
\begin{lstlisting}[language={[LaTeX]Tex}]
\begin{figure}
	\centering
	% \caption antes da figura
	\caption{Leão do site CTAN estudando \TeX} 
	\label{fig:leaoCTAN}
	\includegraphics[width=0.6\textwidth]{imgs/ctanlion}
	\fonte{Duane Bibby, diponibilizador por \url{www.ctan.org}}
\end{figure}
\end{lstlisting}

\begin{figure}[hb!]
	\centering
	\caption{Leão do site CTAN estudando \TeX}
	\label{fig:leaoCTAN}
	\includegraphics[width=0.6\textwidth]{imgs/ctanlion}
	\fonte{Duane Bibby, diponibilizado em \url{www.ctan.org/lion}}
\end{figure}




\section{Quadros e Tabelas}

Tabelas e quadros são elementos tabulares com conteúdos e formatos específicos. 

Tabelas envolvem principalmente números, utiliza-se linhas horizontais somente no topo, final e cabeçalhos e linhas verticais somente nos cabeçalhos. Para reduzir o tamanho do código dos cabeçalhos, o template dispõe do comando \texttt{linhadir}, que adiciona uma linha vertical à célula. Abaixo mostra-se um exemplo do código para a \autoref{tab:exemplo2}:

\begin{lstlisting}[language={[LaTeX]Tex}]
\begin{table}[h]
	\centering
	\caption{Exemplo de Tabela} 
	\label{tab:exemplo2}
	\begin{tabular}{c c c c }
		\hline
		\linhadir{Pessoa}& \linhadir{Livros}& \linhadir{Artigos}& Palestras\\
		\hline
		$p_1$& 1& 3& 4\\
		$p_2$& 1& 3& 3\\
		$p_3$& 1& 3& 4\\
		$p_4$& 3& 5& 2\\
		\hline 
	\end{tabular}
	\vspace{0.3cm}
	\fonte{original} %Fonte do tabela
\end{table}
\end{lstlisting}

\begin{table}[h]
	\centering
	\caption{Exemplo de Tabela} 
	\label{tab:exemplo2}
	\begin{tabular}{c c c c }
		\hline
		\linhadir{\bf Pessoa} & \linhadir{\bf Livros} & \linhadir{\bf Artigos} & \bf Palestras \\
		\hline
		$p_1$ & 1 & 3 & 4 \\
		$p_2$ & 1 & 3 & 3 \\
		$p_3$ & 1 & 3 & 4 \\
		$p_4$ & 3 & 5 & 2 \\
		\hline 
	\end{tabular}
	\vspace{0.3cm}
	\fonte{original} %Fonte do tabela
\end{table}


Quadros se diferem das tabelas por conterem principalmente dados textuais e suas células serem completamente fechadas. O \autoref{quad:exemplo} é um exemplo de quadro.

\begin{quadro}[h]
\centering
\caption{Opiniões sobre esse template}\label{quad:exemplo}
  \begin{tabular}{|l|p{9cm}|}
    \hline 
    \rowcolor[gray]{.9}
    \bf Nome& \bf Opinião\\
    \hline
    Jão& Desculpe, não posso comentar sobre isso.\\
    \hline
    Joana& Literalmente uma revolução do cinema nacional!\\
    \hline
    Jacquin& Esse autor é a vergonha da profisson!\\
    \hline
    Meu cachorro& Au! Au! Au! $\sim$Sons de papel sendo rasgado.\\
    \hline
    Overleaf& ASSINE, ASSINE O PREMIUM.\\
    \hline
    \end{tabular}
    
    \vspace{0.3cm}
	\fonte{original}
\end{quadro}


\section{Padrão das Referências}

Desde que o padrão descrito na \autoref{sec:Refs} seja seguido e partes importantes do template sejam mantidas, as referências não devem ser um problema. O template já vem configurado para o formato padrão da ABNT, desde que os arquivos \texttt{abntex.*}, o preâmbulo e a chamada dos comandos \texttt{bibliographystyle}, \texttt{citeoption}, \texttt{refencias} e \texttt{bibliography}, próximos ao final do arquivo principal, sejam mantidos.

É comum referências serem um problema e, apesar do \LaTeX\ ajudar muito nisso, o processo ainda pode ser complicado. O pacote \texttt{abntex2} está desatualizado, as correções precisaram ser \emph{hard-coded} e o arquivo principal reflete isso.

\textbf{Então, se você quer que as citações e referências sejam tão simples quanto adicionar um \texttt{bibtex} e usar o comando \texttt{cite}, EU TE SUPLICO, não altere o preâmbulo nem os comandos relacionados às referências no \texttt{tempulfa\_main.tex}.}

Referências sortidas para contribuir para a lista ao final (ignore): \cite{Eco1996,Booth2000,BIB2010,Hexsel2004,Franca2001,Gil2002,Porto2002,Silva2005,UFLA:2015,Moura1998,NBR6023:2002,LeGuin:1987}


\chapter{LISTAS, GLOSSÁRIO E ÍNDICE}\label{sec:listasEGlossario}

O template também inclui a criação dos elementos pré-textuais lista de abreviaturas, siglas e símbolos. Além disso, também inclui a criação do glossário e índice.

Todos esses novos elementos, exceto o índice, utilizam o mesmo pacote, \texttt{glossaries}, então, têm comportamento semelhante. Cada ítem deve ser adicionado aos glossários no preâmbulo e, após adicionados, podem ser referenciados no documento com o comando \texttt{gls}. Por exemplo:
	
\begin{figure}[!htb]
	\centering
	\caption{Inserindo ítem no \gls{glo:glossario} e o referindo no texto} %legenda
	\label{fig:exemploglossario1} %rotulo para refencia
	\begin{lstlisting}[language=tex]
\newglossaryentry{glo:glossario}{
	name={glossário}, 
	description={
		Relação de palavras ou expressões técnicas 
		de uso restrito ou de sentido obscuro, 
		utilizadas no texto, acompanhadas das 
		respectivas definições}
	}
		
%\makeglossaries % não é necessario nesse template pois já é chamado na classe
\begin{document}
...
\caption{Inserindo ítem no gls(glo:glossario) e o referindo no texto}
	\end{lstlisting}
	
	\fonte{original}
\end{figure}

O comando \textit{\textbf{gls}} escreve o nome do ítem quando é usado. Para que outro valor seja escrito, ou até para que nada seja escrtio, pode-se utilizar o comando \texttt{glslink}:
\begin{lstlisting}[language=tex]
	\glslink{<rótulo>}{<texto alternativo>}
\end{lstlisting}

É recomendável separar as adições aos glossários em arquivos para melhor organização e reduzir o tamanho do preâmbulo. Nesse template, cada glossário tem seu próprio arquivo, adicionado ao projeto via comandos \texttt{loadglsentries}, e os arquivos têm sua própria pasta. Essa organização pode ser alterada, desde que a mudança seja refletida no preâmbulo.

\newpage
Abaixo listamos o formato para a adição de ítens em cada glossário:

\begin{alineas}
	\item \textbf{abreviaturas}: adicionar uma abreviatura é simples e a sintaxe é relativamente fixa.\\ Um exemplo para a abreviatura de \gls{jan}:
		\begin{lstlisting}[language=tex]
\newabbreviation{jan} % rótulo
{jan.}% forma abreviada
{Janeiro} % forma completa
...
... Um exemplo para a abreviatura de \gls{jan}:
		\end{lstlisting}
	
	\item \textbf{siglas}: adicionar siglas é igualmente simples.\\ 
	Um exemplo para a sigla \gls{abnt}:
	\begin{lstlisting}[language=tex]
\newacronym{abnt} % rótulo
{ABNT} % sigla
{Associação Brasileira de Normas Técnicas} % nome completo
...
... Um exemplo para a sigla \gls{abnt}:
	\end{lstlisting}
	
	\item \textbf{símbolos}: adicionar símbolos é um pouco mais complicado, já que necessita de uma descrição. Como a lista é por ordem de uso, \gls{gama} deve aparecer antes de \gls{alfa}\\ 
	Um exemplo para o símbolo \gls{gama}:
	\begin{lstlisting}[language=tex]
\newglossaryentry{gama}{
	name={$\gamma$}, % o símbolo em questão
	description={Um número gama}, % descrição do símbolo
	type={symbols}} % indicador que é um símbolo
...
... Um exemplo para o símbolo \gls{gama}:
	\end{lstlisting}
	
	\item \textbf{glossário}: adicionar termos no glossário é igual ao exemplo mostrado antes.\\
	Nesse caso, no entanto, adicionamos uma \emph{tag} \LaTeX\ antes do nome, então temos que adicionar o valor "\emph{sort}", para que a ordenação seja correta.
	Um exemplo para o termo \gls{llm}:
	\begin{lstlisting}[language=tex]
\newglossaryentry{llm}{
	name={\emph{large language model}}, % o nome do termo
	description={Grande model de 
		linguagem. Ex:DeepSeek-R1}, % descrição do termo
	sort={large language model}} % chave para ordenação
...
... Um exemplo para o termo \gls{llm}:
	\end{lstlisting}
	
	\item \textbf{índice}: adicionar termos ao índice é o mais fácil de todos. Para adicionar um índice, basta utilizar o comando \emph{index}. Caso algum índice tiver um pai(ou mãe), basta adicionar o delimitador ``!''.\\
	Um exemplo para os termos conjunto\index{Conjunto}, conjunto aberto\index{Conjunto!aberto} e conjunto fechado\index{Conjunto!fechado}:
	\begin{lstlisting}[language=tex]
...Um exemplo para os termos conjunto\index{Conjunto},
conjunto aberto\index{Conjunto!aberto} e conjunto 
fechado\index{Conjunto!fechado}:
	\end{lstlisting}
	
\end{alineas}

Para mais informações sobre como usar esses pacotes, consulte as documentações oficiais do \texttt{glossaries}\footnote{https://ctan.org/pkg/glossaries?lang=en} e do \texttt{imakeidx}\footnote{https://ctan.org/pkg/imakeidx}.

%\include{seções/referencias}
\chapter{\MakeUppercase{Conclusão}}\label{sec:conclusao}

O objetivo deste documento foi apresentar o uso básico da classe \texttt{templufla} para a elaboração de trabalhos acadêmicos da UFLA utilizando \LaTeX. Após edição em \LaTeX, o usuário pode gerar arquivos PDF \cite{PDF2004} ou PostScript \cite{PostScript1999} com grande facilidade.



%==============================================================================
% Incluindo bibliografia
%\bibliographystyle{plain}             % estilo para labels em numeros
%\bibliographystyle{alpha}             % estilo para labels em iniciais
\bibliographystyle{abntex2-alf}           % estilo para referências usando ABNT, 
                                       % precisa instalar o abntex para usar!!!


% precisamos criar opções aqui, já que sobreescrevemos o estilo.
\citeoption{abnt-etal-text=it}
\citeoption{abnt-etal-cite=3}
\citeoption{abnt-missing-year=sd}
\citeoption{abnt-nbr10520=1988} % A opção abnt-cite-style=AuthorYEAR está quebrada
									% essa é equivalente

%inclui Referências Bibliográficas
%inclui Referências Bibliográficas
\referencias
\bibliography{refbib}			% arquivo exemplo refbib.bib
%==============================================================================
% Incluindo glossário

\glossario
%==============================================================================
% Incluindo anexos num1erados com letras maiusculas.
%\apendices

\anexo{Lorem Ipsum}\label{anex:lorem1}

Lorem ipsum dolor sit amet, consectetur adipiscing elit. Vivamus semper, libero egestas pellentesque vulputate, velit felis commodo ante, vel bibendum velit turpis eu felis. Donec viverra quam nisi, vel tincidunt enim tristique interdum. Integer tincidunt a lectus vel porttitor. Nulla venenatis vitae enim ut semper. Nunc in sagittis massa, sit amet dapibus quam. Mauris cursus, ligula ac pretium imperdiet, lectus libero egestas mi, quis tristique leo urna nec erat. In vitae dui maximus, imperdiet massa euismod, auctor enim. Morbi urna odio, accumsan quis magna id, fringilla gravida purus. Aenean facilisis est nisi, nec porttitor purus ullamcorper ut. Proin ac risus congue, aliquet elit in, cursus est.

Vivamus lorem diam, molestie ut ultrices at, feugiat quis tortor. Mauris feugiat, augue at molestie malesuada, purus erat sagittis tellus, sit amet posuere lacus nisl non eros. Sed enim justo, sagittis id elementum quis, commodo ut nibh. Aenean mauris odio, efficitur vel purus sit amet, molestie pharetra arcu. Nunc vel eros sodales, aliquam diam eu, rutrum nisi. Morbi non scelerisque diam. Suspendisse sed dapibus mi, ut sagittis nunc. Praesent ornare, est in rutrum dapibus, tortor massa ornare dolor, at ullamcorper metus augue et ipsum. Sed ut nulla in dolor aliquet faucibus. Quisque rhoncus auctor tellus eu lobortis. Proin rhoncus nisi sit amet nibh tempor hendrerit.\footnote{Lorem ipsum dolor sit amet, consectetur adipiscing elit. Vivamus semper, libero egestas pellentesque vulputate, velit felis commodo ante, vel bibendum velit turpis eu felis. Donec viverra quam nisi, vel tincidunt enim tristique interdum. Integer tincidunt a lectus vel porttitor.}
\include{apendices/apendice1}
%==============================================================================
% Incluindo índice
\indice
%==============================================================================
% Fim do texto
\end{document}
